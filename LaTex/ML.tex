\documentclass[a4paper,twoside,12pt]{report}

\usepackage{graphicx} 
\usepackage{amsmath}

\begin{document}

\pagestyle{empty} 


\title{Machine Learning Project For M2 EKAP}
\author{Laurianne, Lucas}
\date{01/10/2020}

\begin{titlepage}
\flushleft
\includegraphics[width = 4cm, height = 2.5cm]{Consignes/logo un2012quadri_larg40}
\hfill
\includegraphics[width = 4cm, height = 2.5cm]{Consignes/Logo_IAE_vertical}

   \begin{center}
       \vspace*{1cm}

			 \Large
       \textbf{Machine Learning project}

       \vfill
			
			 \vspace{0.7cm}

			 \normalsize
       Aurouet Lucas Moriceau Laurianne
			
			\vspace{0.5cm}
        
			 \normalsize
       A thesis presented for the Master's degree of\\
       Applied Econometrics, under the direction of Pr. B.Sevi
            
       \vspace{0.7cm}
       
       \normalsize   
       Master EKAP\\
       IAE Nantes\\
       31/05/2020
            
   \end{center}
\end{titlepage}

\maketitle

\tableofcontents 
\cleardoublepage 

\pagestyle{plain} 

\chapter{Some small hints}\label{hints}

\section{German Umlauts and other Language Specific Characters}\label{umlauts}
You can type german umlauts like '�', '�', or '�' directly in this file.
This is also true for other language specific characters like '�', '�' etc.

There are problems with automatic hyphenation when using language
specific characters and OT1-encoded fonts. In this case, use a
T1-encoded Type1-font like the Latin Modern font family (\verb#\usepackage{lmodern}#).


\section{References}\label{references}
Using the commands \verb#\label{name}# and \verb#\ref{name}# you are able
to use references in your document. Advantage: You do not need to think
about numerations, because \LaTeX\ is doing that for you.

For example, in section \ref{dividing} on page \pageref{dividing} hints for
dividing large documents are given.

Certainly, references do also work for tables, figures, formulas\ldots

Please notice, that \LaTeX\ usually needs more than one run (mostly 2) to
resolve those references correctly.


\section{Dividing Large Documents}\label{dividing}
You can divide your \LaTeX-Document into an arbitrary number of \TeX-Files
to avoid too big and therefore unhandy files (e.g. one file for every chapter).

For this, you insert in your main file (this one) for every subfile
the command '\verb#\input{subfile}#'. This leads to the same behavior
as if the content of the subfile would be at the place of the \verb#\input#-Command.

%% <== End of hints
%%%%%%%%%%%%%%%%%%%%%%%%%%%%%%%%%%%%%%%%%%%%%%%%%%%%%%%%%%%%%



%%%%%%%%%%%%%%%%%%%%%%%%%%%%%%%%%%%%%%%%%%%%%%%%%%%%%%%%%%%%%
%% BIBLIOGRAPHY AND OTHER LISTS
%%%%%%%%%%%%%%%%%%%%%%%%%%%%%%%%%%%%%%%%%%%%%%%%%%%%%%%%%%%%%
%% A small distance to the other stuff in the table of contents (toc)
\addtocontents{toc}{\protect\vspace*{\baselineskip}}

%% The Bibliography
%% ==> You need a file 'literature.bib' for this.
%% ==> You need to run BibTeX for this (Project | Properties... | Uses BibTeX)
%\addcontentsline{toc}{chapter}{Bibliography} %'Bibliography' into toc
%\nocite{*} %Even non-cited BibTeX-Entries will be shown.
%\bibliographystyle{alpha} %Style of Bibliography: plain / apalike / amsalpha / ...
%\bibliography{literature} %You need a file 'literature.bib' for this.

%% The List of Figures
\clearpage
\addcontentsline{toc}{chapter}{List of Figures}
\listoffigures

%% The List of Tables
\clearpage
\addcontentsline{toc}{chapter}{List of Tables}
\listoftables


%%%%%%%%%%%%%%%%%%%%%%%%%%%%%%%%%%%%%%%%%%%%%%%%%%%%%%%%%%%%%
%% APPENDICES
%%%%%%%%%%%%%%%%%%%%%%%%%%%%%%%%%%%%%%%%%%%%%%%%%%%%%%%%%%%%%
\appendix
%% ==> Write your text here or include other files.

%\input{FileName} %You need a file 'FileName.tex' for this.


\end{document}

